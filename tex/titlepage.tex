%% Titlepage

%%%% SUBFILE %%%%
% subfiles allows to compile from a subfile and not main document
% to make it work in a folder structure, the root document needs to be specified
% in each subfile with the following line (placed somewhere at the beginning):
% !TEX root =  ../document.tex

	\addchap*{\hspace*{4.2cm} {\huge Abschlussbericht}}
	%\vspace*{3cm}
	\thispagestyle{empty}
	\vspace*{-5.3cm}
	HTWK Leipzig \\
	Fakultät Elektrotechnik und Informationstechnik \\

	\vspace*{-2cm}% Logo nach oben verschieben
	\makebox[\textwidth][r]{%
	\includegraphics[scale=0.1]{img/HTWK-Logo}% Logo einfügen
	}\par
	\vspace*{3.5cm}% Abstand
	\begin{center}
	%	zu den Lehrveranstaltungen\\
		\vspace*{0.3cm}% Abstand
	\textbf{Praxisforschungsprojekt}\\
	%\textbf{8516 Embedded Systems II} (Sommersemester 2015)\\
	%Sommersemester 2014
	\end{center}
	\ \\
	\vspace*{1cm}% Abstand
	\begin{center}
	\LARGE{Integration einer Roboterfertigungszelle in den Fahrzeugherstellungsprozess}
	\end{center}
	\vspace*{3cm}% Abstand
	%\uline{Aufgabe}\\
	%\ \\
	%In diesem Komplexpraktikum ist eine \textit{AS-Interface} Anlage in Betrieb zu nehmen und zu programmieren. Als Anlage wird dabei ein Modellriesenrad mit diversen Sicherheitseinrichtungen und Peripheriegeräten dienen, die über das Bussystem \textit{AS-Interface} angeschlossen sind.\\
\begin{spacing}{1.3}
	\begin{tabbing}
	Hochschulbetreuer:\hspace*{2.5cm} \= Prof. Dr.-Ing. Andreas Pretschner\\
	Betriebl. Betreuer:\>Elias Froschauer\\
	\ \\
	Vorgelegt von:\> Ronny Döring\\
	%\> Thomas Bauer\\
	\ \\
	Studiengang:\> Elektrotechnik und Informationstechnik\\
	Seminargruppe:\> 13EIM-AT\\
	Matrikelnummer:\> 62308\\
	%\> 61854\\
	\ \\
	Emailadresse:\> \href{mailto:RonnyDoering.Ing@gmail.com}{RonnyDoering.Ing@gmail.com}\\
	%\> \href{mailto:baui.leipzig@gmail.com}{baui.leipzig@gmail.com}\\
	\ \\
	Datum der Abgabe:\> dd.mm.2016 \\
	\end{tabbing}
	%\end{raggedright}

\end{spacing}

	%\newpage

	%\blindtext
