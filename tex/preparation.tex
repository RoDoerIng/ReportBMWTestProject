%% Roadmap

%%%% SUBFILE %%%%
% subfiles allows to compile from a subfile and not main document
% to make it work in a folder structure, the root document needs to be specified
% in each subfile with the following line (placed somewhere at the beginning):
% !TEX root =  ../document.tex

\bild{t}{trim = 0cm 1.2cm 0cm 1.2cm, clip, width=0.8\textwidth}{roadmap}{Anlagenentstehungsprozess: Roadmap}

%\section{Roadmap}
\section{Vorbereitung}
\label{sec:preparation}

Der Anlagenentstehungsprozess ist an eine im Standard enthaltene Roadmap geknüpft. Abbildung~\ref{fig:roadmap} zeigt eine gegliederte Darstellung dieser Roadmap in etwas vereinfachter Weise. %Wie bereits in Kapitel~\ref{chap:intro} erwähnt, sind die einzelnen Schritte während der gesamten Projektbearbeitung klar definiert.
Der gesamte Prozess lässt sich in sechs Abschnitte gliedern. Jeder Abschnitt enthält wiederum verschiedene Schritte. Ein neuer Schritt wird erst abgearbeitet, wenn der vorangegangene abgeschlossen und durch BMW abgenommen ist.

Vor der eigentlichen Bearbeitung des Projektumfanges erhält der Anlagenhersteller eine aktuelle Version der Dokumente, die den Standard bilden. Das sind Dokumente folgender Kategorien:
%\begin{spacing}{0.5}
  \begin{itemize}
    \itemsep0.1em
    \item Kaufteile
    \item Planung
    \item EPLAN
    \item Hardware Konstruktion
    \item Antriebstechnik
    \item IT-System Schnittstellen
    \item Roboter
    \item Visualisierung mit TIA-WinCC
    \item Classic SPS Siemens SIAMTIC
    \item Dokumentation
    \item Fördertechnik
    \item Allgemeines
  \end{itemize}

%\end{spacing}
In seiner Gesamtheit umfasst der TMO V1 Standard über 250 Dokumente, die zu einem großen Teil im Excel-, Word- oder PDF-Format vorliegen. Neben einer ganzen Reihe an Richtlinien, steht eine Vielzahl an fertigen Projektvorlagen zur Verfügung. Diese erleichtern die spätere Bearbeitung und sorgen dafür, dass alle Projekte auf einer gleichen Basis aufbauen.

Nach dem der Angebotsprozess abgeschlossen ist und die Formalitäten zwischen Anlagenbetreiber und Hersteller geklärt sind, findet schließlich der Kick Off auf Leitungsebene und anschließend auf technischer Ebene statt.
