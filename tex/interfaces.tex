%% Interfaces and Communication

%%%% SUBFILE %%%%
% subfiles allows to compile from a subfile and not main document
% to make it work in a folder structure, the root document needs to be specified
% in each subfile with the following line (placed somewhere at the beginning):
% !TEX root =  ../document.tex

\chapter{Konzept zur Anbindung der IPS-X Systeme}
\label{chap:interfaces}

\section{IPS-L}
\label{sec:ips_l}

\subsection{Schnittstellenkontrakt}
\label{subsec:ips_l_contract}

Das IPS-L-Netz ist Bestandteil des übergeordneten IT-Netzes der Anlage. Es dient zum Austausch von Daten die die Logistik-Steuerung der Auftragsdaten betreffen. Dazu verwaltet das IPS-L-System Karossen und Anbauteile und stellt diese Informationen entsprechend der Anlage bereit. Dabei entspricht die Reihenfolge der Daten, die vom IPS-L-System an die Steuerung übermittelt werden, der Sequenz an Fahrzeugen, die in der Anlage bearbeitet werden. Somit können die automatischen Bearbeitungsprozesse in Abhängigkeit des vorliegenden Fahrzeuges angepasst werden.

Im Schnittstellenkontrakt mit BMW wurde zu Projektbeginn festgelegt, dass es sich bei der Schlüsselnummer der Datensätze vom IPS-L-System um die Auftragsnummer handelt. Zudem wurden gemäß Abbildung~\ref{fig:ips_l_telegrams} folgende vier Telegramme für den Datenaustausch spezifiziert:

\begin{description}
    \item [Einzeldatensatz-Anforderung] Die SPS sendet ein Einzel-Anforderungstelegramm an das IPS-L-System und dieses antwortet mit einem einzelnen Datensatz
    \item [Blockdatensatz-Anforderung] Die SPS sendet ein Block-Anforderungstelegramm an das IPS-L-System und dieses antwortet mit einem Block-Datensatz dessen Größe auf fünf Einzeldatensätze festgelegt wurde
    \item [IPS-L Verbindungsüberwachung] Die SPS sendet einen Ping an das IPS-L-System und dieses antwortet mit einer Ping-Antwort
    \item [SPS Verbindungsüberwachung] Das IPS-L-System sendet einen Ping an die SPS und diese antwortet mit einer Ping-Antwort
\end{description}

\bild{t}{width=0.8\textwidth}{ips_l_telegrams}{IPS-L Schnittstellenkontrakt}

\subsection{Realisierungskonzept}
\label{subsec:ips_l_concept}


\section{IPS-T}
\label{sec:ips_t}

\subsection{Schnittstellenkontrakt}
\label{subsec:ips_t_contract}

 Es handelt sich dabei um das Technik-Netz. Mit diesem werden Daten unterscheidlicher Art ausgetauscht. Zu den wichtigsten Daten gehören die Störmeldungen, welche die SPS an das IPS-T-Netz übermittelt. Zusätzlich werden Daten zu Zählerständen, Füllständen und Taktzeiten ausgetauscht. Dabei agiert die SPS nicht nur als Sender sondern auch Empfänger. So können mittels IPS-T beispielsweise Sollwerte für den Prozess  vorgegeben, bit-getriggerte Events in der Anlage ausgelöst oder Schichtwechsel-Signale an die Steuerung gesendet werden. 


Zum Generieren und Interpretieren der Daten bedienen sich die Kommunikationspartner eines im TMO-V1 definierten Protokolls. Seitens der SPS ist dieses in Form eines STRUCT in den Sende- und Empfangs-DBs hinterlegt (siehe Abb.~\ref{fig:ips_t_struct}). Das Befüllen der STRUCTs in den DBs geschieht mittels BMW-eigener Funktionen und Funktionsbausteinen, die wiederum separate DBs als Datenbasis nutzen. Die Aufgabe des SPS-Entwicklers besteht darin die Datenbasis mit den relevanten Werten zu befüllen. Im vorliegenden Projekt ist der Austausch von insgesamt sechs Zählerständen vorgesehen, die die folgenden Daten enthalten sollen: 

\begin{description}
    \itemsep 0.05em
    \item [IPS-T -> SPS] Sollwert Taktzeit (Einheit: \einh{1/10}{\second})
    \item [IPS-T -> SPS] Schichtziel (Einheit: \einh{}{\cars})
    \item [SPS -> IPS-T] Gesamtzahl Autos (Einheit: \einh{}{\cars})
    \item [SPS -> IPS-T] Aktuelle Taktzeit (Einheit: \einh{1/10}{\second})
    \item [SPS -> IPS-T] Gesamtzahl F20 (Einheit: \einh{}{\cars})
    \item [SPS -> IPS-T] Gesamtzahl F45 (Einheit: \einh{}{\cars})
\end{description}

\bild{t}{width=0.8\textwidth}{ips_t_struct}{STRUCT: IPS-T Protokoll}

\subsection{Realisierungskonzept}
\label{subsec:ips_l_concept}

\begin{enumerate}
    \item GetBaseData via pointer
    \item UDTJOBTYPE  GetJob from type management
    \item Process UDTJOBTYPE
\end{enumerate}


\section{IPS-Q}
\label{sec:ips_q}

Das Qualitätsnetz IPS-Q hat im Rahmen des Testprojektes keines Relevanz und wird aus diesem Grund nicht konzeptioniert.