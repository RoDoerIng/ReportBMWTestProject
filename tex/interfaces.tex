%% Interfaces and Communication

%%%% SUBFILE %%%%
% subfiles allows to compile from a subfile and not main document
% to make it work in a folder structure, the root document needs to be specified
% in each subfile with the following line (placed somewhere at the beginning):
% !TEX root =  ../document.tex

\chapter{Nutzung der IPS-X Systeme}
\label{chap:interfaces}

\section{IPS-L}
\label{sec:ips_l}

\subsection{Schnittstellenkontrakt}
\label{subsec:ips_l_contract}

Das IPS-L-Netz ist Bestandteil des übergeordneten IT-Netzes der Anlage. Es dient zum Austausch von Daten die die Logistik-Steuerung der Auftragsdaten betreffen. Dazu verwaltet das IPS-L-System Karossen und Anbauteile und stellt diese Informationen entsprechend der Anlage bereit. Dabei entspricht die Reihenfolge der Daten, die vom IPS-L-System an die Steuerung übermittelt werden, der Sequenz an Fahrzeugen, die in der Anlage bearbeitet werden. Somit können die automatischen Bearbeitungsprozesse in Abhängigkeit des vorliegenden Fahrzeuges angepasst werden.

Im Schnittstellenkontrakt mit BMW wurde zu Projektbeginn festgelegt, dass es sich bei der Schlüsselnummer der Datensätze vom IPS-L-System um die Ordernummer handelt. Zudem wurden gemäß Abbildung~\ref{fig:ips_l_telegrams} folgende vier Telegramme für den Datenaustausch spezifiziert:

\begin{description}
    \item [Einzeldatensatz-Anforderung] Die SPS sendet ein Einzel-Anforderungstelegramm an das IPS-L-System und dieses antwortet mit einem einzelnen Datensatz
    \item [Blockdatensatz-Anforderung] Die SPS sendet ein Block-Anforderungstelegramm an das IPS-L-System und dieses antwortet mit einem Block-Datensatz dessen Größe auf fünf Einzeldatensätze festgelegt wurde
    \item [IPS-L Verbindungsüberwachung] Die SPS sendet einen Ping an das IPS-L-System und dieses antwortet mit einer Ping-Antwort
    \item [SPS Verbindungsüberwachung] Das IPS-L-System sendet einen Ping an die SPS und diese antwortet mit einer Ping-Antwort
\end{description}

\bild{t}{width=0.8\textwidth}{ips_l_telegrams}{IPS-L Schnittstellenkontrakt}
% \newpage
\subsection{Realisierungskonzept}
\label{subsec:ips_l_concept}

Die Daten aus dem IPS-L-System sollen in der Typverwaltung zu Anwendung kommen. Wie in Abschnitt~\ref{sec:project_content} beschrieben, soll anhand des in der Station befindlichen Fahrzeugtyps ein entsprechendes Bearbeitungsprofil ausgewählt werden. Zur Identifikation des Fahrezeugtypes wird dessen Entwicklungsserie (z.B. E-90, E-87) verwendet.


Für die Typverwaltung ist ein eigener Funktionsbaustein (FB\abbrev{FB}{Funktionsbaustein}) \texttt{FB\_TYPE\_ADMIN} vorgesehen, der zyklisch im Hauptprogramm (OB1\abbrev{OB}{Organisationsbaustein}) der SPS aufgerufen werden soll. Dessen Logik ist konzeptionell in dem Sequenzdiagramm in Abbildung~\ref{fig:ips_l_concept_sequence} dargestellt und soll auf Basis dieser Vorlage umgesetzt werden. Folgende Funktionalität ist dabei zu implementieren. 
Zunächst wird auf die Meldung der Fördertechnik gewartet, dass der Werkstückträger (WT\abbrev{WT}{Werkstückträger}) in Bearbeitungsposition ist. Außerdem wird geprüft, ob die Bearbeitung nicht aktiv ist. Sind beide Bedingungen erfüllt, wird die Nummer des WTs von Förderelement \texttt{+TL002} aus dem Datenbaustein (DB\abbrev{DB}{Datenbaustein}) \texttt{DB\_WT\_NR} gelesen. Anhand dieser Nummer wird anschließend aus \texttt{DB\_WT\_DA} die auf dem WT befindliche Ordernummer ermittelt. Mithilfe dieser wird danach die zur Ordenummern gehörige Entwicklungsserie aus dem IPS-L DB \texttt{DB\_IPSL\_SEQU} gelesen. Der DB \texttt{DB\_TYPE\_ADMIN} enthält schließlich die Zuordnung der Bearbeitungsprofile zu den Entwicklungsserien. Das zugehörige Bearbeitungsprofil wird also ausgewählt und an die Roboterschnittstelle \texttt{DB\_ROB\_ITF} übergeben. Nach erfolgter Übergabe erhält der Roboter die Freigabe zur Bearbeitung, er meldet zurück, sobald die Bearbeitung aktiv ist. 

\bild{t}{width=\textwidth}{ips_l_concept_sequence}{Sequenzdiagramm: Verwendung der IPS-L-Daten zur Typverwaltung}



% https://mermaidjs.github.io/mermaid-live-editor/#/view/c2VxdWVuY2VEaWFncmFtCiAgICBsb29wIGplZGVuIFp5a2x1cwogICAgICAgIGFsdCAgSXN0IEZhaHJ6ZXVnIGluIEJlYXJiZWl0dW5nc3Bvc2l0aW9uIFVORCBOSUNIVCBCZWFyYmVpdHVuZyBha3RpdgogICAgICAgICAgICBGQl9UWVBFX0FETUlOLT4-K0RCX1dUX05SOiBXVC1Oci4gYW4gK1RMMDAyIGxlc2VuCiAgICAgICAgICAgIERCX1dUX05SIC0tPj4tRkJfVFlQRV9BRE1JTjogV1QtTnIuCiAgICAgICAgICAgIEZCX1RZUEVfQURNSU4tPj4rREJfV1RfREE6IE9yZGVybnVtbWVyIGxlc2VuCiAgICAgICAgICAgIERCX1dUX0RBLS0-Pi1GQl9UWVBFX0FETUlOOiBPcmRlcm51bW1lcgogICAgICAgICAgICBGQl9UWVBFX0FETUlOLT4-K0RCX0lQU0xfU0VROiBFbnR3aWNrbHVuZ3NzZXJpZSBsZXNlbgogICAgICAgICAgICBEQl9JUFNMX1NFUS0tPj4tRkJfVFlQRV9BRE1JTjogRW50d2lja2x1bmdzc2VyaWUKICAgICAgICAgICAgRkJfVFlQRV9BRE1JTi0-PitEQl9UWVBFX0FETUlOOiBCZWFyYmVpdHVuZ3Nwcm9maWwgbGVzZW4KICAgICAgICAgICAgREJfVFlQRV9BRE1JTi0tPj4tRkJfVFlQRV9BRE1JTjogQmVhcmJlaXR1bmdzcHJvZmlsCiAgICAgICAgICAgIEZCX1RZUEVfQURNSU4tPj4rREJfUk9CX0lURjogQmVhcmJlaXR1bmdzcHJvZmlsIGFuIFJvYm90ZXIgw7xiZXJnZWJlbgogICAgICAgICAgICBGQl9UWVBFX0FETUlOLT4-REJfUk9CX0lURjogRnJlaWdhYmUgQmVhcmJlaXR1bmcKICAgICAgICAgICAgREJfUk9CX0lURi0tPj4tRkJfVFlQRV9BRE1JTjogQmVhcmJlaXR1bmcgYWt0aXYKICAgICAgICBlbHNlCiAgICAgICAgICAgIE5vdGUgcmlnaHQgb2YgRkJfVFlQRV9BRE1JTjoga2VpbmUgSW50ZXJha3Rpb24KICAgICAgICBlbmQKICAgIGVuZA

% % MERMAID SEQUENCE DIAGRAM text
% ```mermaid
% sequenceDiagram
%     loop jeden Zyklus
%         alt  Ist Fahrzeug in Bearbeitungsposition UND NICHT Bearbeitung aktiv
%             FB_TYPE_ADMIN->>+DB_WT_NR: WT-Nr. an +TL002 lesen
%             DB_WT_NR -->>-FB_TYPE_ADMIN: WT-Nr.
%             FB_TYPE_ADMIN->>+DB_WT_DA: Ordernummer lesen
%             DB_WT_DA-->>-FB_TYPE_ADMIN: Ordernummer
%             FB_TYPE_ADMIN->>+DB_IPSL_SEQ: Entwicklungsserie lesen
%             DB_IPSL_SEQ-->>-FB_TYPE_ADMIN: Entwicklungsserie
%             FB_TYPE_ADMIN->>+DB_TYPE_ADMIN: Bearbeitungsprofil lesen
%             DB_TYPE_ADMIN-->>-FB_TYPE_ADMIN: Bearbeitungsprofil
%             FB_TYPE_ADMIN->>+DB_ROB_ITF: Bearbeitungsprofil an Roboter übergeben
%             FB_TYPE_ADMIN->>DB_ROB_ITF: Freigabe Bearbeitung
%             DB_ROB_ITF-->>-FB_TYPE_ADMIN: Bearbeitung aktiv
%         else
%             Note right of FB_TYPE_ADMIN: keine Interaktion
%         end
%     end
% ```


\subsection{Implementierung}
\label{subsec:ips_l_implementation}

Für die Implementierung der beschriebenen Logik werden neben dem Funktionsbaustein \texttt{FB\_TYPE\_ADMIN}, der die Logik enthält, zahlreiche Hilfsfunktionen (FCs\abbrev{FC}{Funktion}), der Datenbaustein \texttt{DB\_TYPE\_ADMIN} sowie ein neuer Datentyp (UDT\abbrev{UDT}{Datentyp (User Defined Type)}) zum Darstellen des Bearbeitungsprofils benötigt. Folgende Auflistung beinhaltet die nötigen Objekte und beschreibt deren Zweck in Anlehnung an Abbildung~\ref{fig:ips_l_concept_sequence}. Zudem enthält die Auflistung die Implementierung der Obejekte in der Sprache \emph{SCL}. Alle anderen im Sequenzdiagramm aufgeführten Datenbausteine, welche nicht beschrieben sind, sind bereits in der STEP7-Projektvorlage vorhanden und können ohne Modifikation verwendet werden. 


% \begin{description}
%     \itemsep 0.05em
%     % \item \texttt{FC\_IPSL\_SET\_POINTER8}
%     \item [\texttt{UDT\_PROCESS\_PROFILE}:] Repräsentation des Bearbeitungsprofils
%     \item [\texttt{FC\_GET\_WT\_NO():INT}:] Lesen der WT-Nummer
%     \item [\texttt{FC\_GET\_OR(wtNumber:INT):CHAR[8]}:] Lesen der Ordernummer
%     \item [\texttt{FC\_GET\_SERIES(orderNumber:CHAR[8]):CHAR[4]}:] Lesen der Entwicklungsserie
%     \item [\texttt{FC\_GET\_PROCESS\_PROFILE(series:CHAR[4]:UDT\_PROCESS\_PROFILE}:] Lesen des Bearbeitungsprofils
%     \item [\texttt{DB\_TYPE\_ADMIN}:] Datenablage für Typverwaltung
%     \item [\texttt{FB\_TYPE\_ADMIN}:] Steuerung der Typverwaltung
% \end{description}


\begin{description}
    \itemsep 0.05em
    % \item \texttt{FC\_IPSL\_SET\_POINTER8}
    \item [\texttt{UDT\_PROCESS\_PROFILE}] ist ein benutzerdefinierter Datentyp, der als Repräsentation für ein Bearbeitungsprofil dient. Die Entwicklungsserie wird einem Feld von vier Elementen des Typs \texttt{CHAR\abbrev{CHAR}{Character (Zeichen) \einh{8}{\Bit}}} dargestellt. Das entspricht exakt der Repräsentation, wie sie im Datenbaustein \texttt{DB\_IPSL\_SEQU} verwendet wird. Daneben enthält der Datentyp für jeden anzusteuernden Zylinder jeweils eine boolesche. Weißt eine \texttt{useCylX}-Variable den Wert \texttt{TRUE} auf, wird der zugehörige Zylinder bei der Bearbeitung des Fahrzeugs angesteuert.\leer
    % 
    \lstinputlisting[style=sclStyle,label=lst:udt_process_profile]{\src/udt_process_profile.scl}
    %
    \item [\texttt{DB\_TYPE\_ADMIN}:] ist ein Datenbaustein, der die verschiedenen Bearbeitungsprofile als ein Feld mit zehn Elementen vom Typ \texttt{UDT\_PROCESS\_PROFILE} enthält. Die enthaltenen Daten werden über das HMI eingepflegt und geändert.\leer
    % 
    \lstinputlisting[style=sclStyle,label=lst:db_type_admin]{\src/db_type_admin.scl}
    %
    \item [\texttt{FC\_GET\_WT\_NO(conveyorElement:INT):INT}] liest die WT-Nummer aus \texttt{DB\_WT\_NR} an einem durch \texttt{conveyorElement} spezifizierten Förderelement und gibt sie als Ganzzahl zurück. \abbrev{INT}{Integer (vorzeichenbehaftete Ganzzahl) \einh{16}{\Bit}} Für \texttt{conveyorElement < 0} wird \texttt{-1} zurückgegeben, um damit einen Fehler zu signalisieren.\leer
    % 
    \lstinputlisting[style=sclStyle,label=lst:fc_get_wt_no]{\src/fc_get_wt_no.scl}
    % 
    \item [\texttt{FC\_GET\_OR(wtNumber:INT, orderNumber:CHAR[8]):VOID}] liest die Ordernummer aus dem Datenbaustein \texttt{DB\_WT\_DA} von einem durch \texttt{wtNumber} spezifizierten Werkstückträger und gibt diese über den In/Out-Parameter \texttt{orderNumber} als ein Feld von acht Elementen des Typs \texttt{CHAR} zurück. Die Funktion selbst ist vom Typ \texttt{VOID}. Die Einschränkungen für Rückgabewerte seitens des SIMATIC Managers erlauben keine Arrays. Das Format für die Ordernummer entspricht dem, welches im DB \texttt{DB\_IPSL\_SEQU} verwendet wird und erleichtert damit die weitere Verwendung.\leer
    %
    \lstinputlisting[style=sclStyle,label=lst:fc_get_or]{\src/fc_get_or.scl}
    %
    \item [\texttt{FC\_GET\_SERIES(orderNumber:CHAR[8], series:CHAR[4]):VOID}] liest unter Angabe der Ordernummer als Parameter \texttt{orderNumber} die Entwicklungsserie aus dem Datenbaustein \texttt{DB\_IPSL\_SEQU} und gibt diese als In/Out-Parameter \texttt{series} als Feld von vier Elementen des Typs \texttt{CHAR} zurück.\par    
    Beim Lesen der IPS-L Daten bedient sich die Funktion des im Datenbaustein \texttt{DB\_IPSL\_VERW} vorhandenen \emph{Zeiger 8}. Dieser ist vom TMO V1 Standard für den einfachen Zugriff auf das IPS-L Sequenzregister definiert und zeigt auf den Datensatz des aktuellen Fahrzeuges. Der Zeigerwert wird von einer separaten Funktion verwaltet. Bevor die Entwicklungsserie aus den IPSL-Daten gelesen wird prüft die Funktion, ob die angeforderte Ordernummer derjenigen entspricht, welche durch den \emph{Zeiger 8} referenziert wird. Ist dies nicht der Fall gibt die Funktion einen Fehler zurück.\leer
    %
    \lstinputlisting[style=sclStyle,label=lst:fc_get_series]{\src/fc_get_series.scl}
    % 
    \item [\texttt{FC\_GET\_PROCESS\_PROFILE(series:CHAR[4]:UDT\_PROCESS\_PROFILE}] liest unter Angabe der Entwicklungsserie als Parameter \texttt{series} das zugehörige Bearbeitungsprofil aus dem Datenbaustein \texttt{DB\_TYPE\_ADMIN} und gibt es als In/Out-Parameter \texttt{processProfile} des Typs \texttt{UDT\_PROCESS\_PROFILE} zurück.
    % 
    \lstinputlisting[style=sclStyle,label=lst:fc_get_process_profile]{\src/fc_get_process_profile.scl}
    % \item [\texttt{FB\_TYPE\_ADMIN}:] implementiert die Logik zur Typverwaltung und ist damit für deren Steuerung verantworlich. Der Aufruf findet zyklisch im OB1 statt. 
\end{description}


\section{IPS-T}
\label{sec:ips_t}

\subsection{Schnittstellenkontrakt}
\label{subsec:ips_t_contract}

 Es handelt sich dabei um das Technik-Netz. Mit diesem werden Daten unterscheidlicher Art ausgetauscht. Zu den wichtigsten Daten gehören die Störmeldungen, welche die SPS an das IPS-T-Netz übermittelt. Zusätzlich werden Daten zu Zählerständen, Füllständen und Taktzeiten ausgetauscht. Dabei agiert die SPS nicht nur als Sender sondern auch Empfänger. So können mittels IPS-T beispielsweise Sollwerte für den Prozess  vorgegeben, bit-getriggerte Events in der Anlage ausgelöst oder Schichtwechsel-Signale an die Steuerung gesendet werden. 


Zum Generieren und Interpretieren der Daten bedienen sich die Kommunikationspartner eines im TMO-V1 definierten Protokolls. Seitens der SPS ist dieses in Form eines STRUCT in den Sende- und Empfangs-DBs hinterlegt (siehe Abb.~\ref{fig:ips_t_struct}). Das Befüllen der STRUCTs in den DBs geschieht mittels BMW-eigener Funktionen und Funktionsbausteinen, die wiederum separate DBs als Datenbasis nutzen. Die Aufgabe des SPS-Entwicklers besteht darin die Datenbasis mit den relevanten Werten zu befüllen. Im vorliegenden Projekt ist der Austausch von insgesamt sechs Zählerständen vorgesehen, die die folgenden Daten enthalten sollen:

\begin{tabbing}
    \textbf{IPS-T}    \= \textbf{--> SPS}   \hspace{.5cm}\= Sollwert Taktzeit (Einheit: \einh{1/10}{\second}) \\
    \textbf{IPS-T}    \> \textbf{--> SPS}   \> Schichtziel (Einheit: \einh{}{\cars}) \\
    \textbf{SPS  }    \> \textbf{--> IPS-T} \> Gesamtzahl Autos (Einheit: \einh{}{\cars})\\
    \textbf{SPS  }    \> \textbf{--> IPS-T} \> Aktuelle Taktzeit (Einheit: \einh{1/10}{\second})\\
    \textbf{SPS  }    \> \textbf{--> IPS-T} \> Gesamtzahl F20 (Einheit: \einh{}{\cars})\\
    \textbf{SPS  }    \> \textbf{--> IPS-T} \> Gesamtzahl F45 (Einheit: \einh{}{\cars})
\end{tabbing}

% \begin{description}
%     \itemsep 0.05em
%     \item [IPS-T -> SPS] Sollwert Taktzeit (Einheit: \einh{1/10}{\second})
%     \item [IPS-T -> SPS] Schichtziel (Einheit: \einh{}{\cars})
%     \item [SPS -> IPS-T] Gesamtzahl Autos (Einheit: \einh{}{\cars})
%     \item [SPS -> IPS-T] Aktuelle Taktzeit (Einheit: \einh{1/10}{\second})
%     \item [SPS -> IPS-T] Gesamtzahl F20 (Einheit: \einh{}{\cars})
%     \item [SPS -> IPS-T] Gesamtzahl F45 (Einheit: \einh{}{\cars})
% \end{description}

\bild{t}{width=0.8\textwidth}{ips_t_struct}{STRUCT: IPS-T Protokoll}

% \subsection{Realisierungskonzept}
% \label{subsec:ips_t_concept}




\section{IPS-Q}
\label{sec:ips_q}

Das Qualitätsnetz IPS-Q hat im Rahmen des Testprojektes keines Relevanz und wird aus diesem Grund nicht konzeptioniert.