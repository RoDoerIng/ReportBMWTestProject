%% Introduction

%%%% SUBFILE %%%%
% subfiles allows to compile from a subfile and not main document
% to make it work in a folder structure, the root document needs to be specified
% in each subfile with the following line (placed somewhere at the beginning):
% !TEX root =  ../document.tex

\chapter{Einleitung}
\label{chap:intro}

\section{Ausgangspunkt}

Die OSB AG ist ein Unternehmen mit deutschlandweit zwölf Niederlassungen und über 500 Mitarbeitern. Der Standort Leipzig ist in den Bereichen der Software und Automation tätig. Zu den Kunden zählen hauptsächlich lokal ansässige Produktionsunternehmen aus verschiedensten Industriezweigen. Dabei spielen die im Leipziger Einzugsgebiet vertretenen Automobilhersteller eine große Rolle.
Das BMW Group Werk Leipzig gehört zu den langjährigen Kunden der OSB AG. Speziell für die Abteilung CFK wurde eine Vielzahl an Projekten durchgeführt.

Den Löwenanteil der BMW Niederlassung in Leipzig bildet jedoch der Technologiebereich Montage. Dieser Bereich ist hochautomatisiert mit verschiedensten Anlagen unterschiedlicher Hersteller und ausgelegt auf einen nahezu unterbrechungsfreien Betrieb. Um eine derartige Komplexität zu beherrschen, stützen sich die Anlagenhersteller auf einen sehr umfangreichen und detaillierten Standard, dem \emph{TMO V1} Standard, welcher eigens von BMW für die Montage entwickelt wurde.

Der genannte Standard stellt einen kompletten Leitfaden für den Anlagenhersteller von Projektbeginn bis zur Übergabe an BMW dar. Damit wird sichergestellt, dass während der gesamten Projektphase die Konformität zum Standard eingehalten wird und bei der späteren Integration der Anlage in die Produktion keine Kompatibilitätsprobleme auftreten.

Um Anlagen herzustellen, zu betreuen oder zu erweitern verlangt BMW von potentiellen Dienstleistern einen Nachweis der Konformität zu diesem Standard. Im Fall der OSB AG am Standort Leipzig besteht dieser Nachweis in der Bearbeitung eines fiktiven Projektes, im Folgenden \emph{Testprojekt} genannt, welches einen Großteil der geforderten Design-Richtlinien abdeckt. Neben der Konformitätsbewertung der OSB AG hinsichtlich des TMO V1 Standards bildet dieses Testprojekt den thematischen Rahmen für diesen Bericht.


\section{Aufgabenstellung}

\textbf{Integration einer Roboterzelle in den Fahrzeugherstellungsprozess}\leer
Teilaufgaben:%\leer
\begin{enumerate}
	\item Einarbeitung in die herstellerspezifischen Richtlinien
	\item Konzeption der Steuerungstechnik
	\item Konzeption interner und externer Schnittstellen
	\item Softwaretechnische Umsetzung
\end{enumerate}
