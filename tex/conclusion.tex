%% Conclusion

%%%% SUBFILE %%%%
% subfiles allows to compile from a subfile and not main document
% to make it work in a folder structure, the root document needs to be specified
% in each subfile with the following line (placed somewhere at the beginning):
% !TEX root =  ../document.tex

\chapter{Fazit}

BMW als weltweit agierender Fahrzeughersteller stellt nicht nur hohe Ansprüche an die Qualität seiner Fahrzeuge. Besonders in der Endmontage beginnt der Qualitätsgedanke bereits bei den Montageanlagen. Um eine durchgehend hohe Qualität bei der Planung und Entwicklung dieser Anlagen zu erzielen, hat BMW einen Standard entwickelt, der standortübergreifend eingesetzt wird und damit ein hohes Maß an Konsistenz und Zuverlässigkeit schafft. Dementsprechend wichtig ist es, dass die Anlagenhersteller und Automatisierungs-Dienstleister den Standard kennen und umsetzen.\par

Der Grundgedanke hinter diesem Praxisforschungsprojekt war es, das Rahmenwerk des Standards kennenzulernen, es anzuwenden und damit produktiv zu werden. Gleichzeitig sollte es dazu dienen, sich gegenüber BMW als standardkonformer Dienstleister aufzustellen, um zukünftig an Ausschreibungen für Anlagenänderungen oder Neuerrichtungen im Bereich der Fahrzeug-Endmontage teilzunehmen.\par

Grundsätzlich ist es gelungen, einen sehr weitreichenden Einblick in die Umfänge des Standards zu erhalten und dessen Komplexität zu erfassen. Es ist außerdem gelungen große Teile der Planungsphase zu durchlaufen und damit eine fundierte Basis für die Entwicklungsarbeit zu schaffen. Schließlich war es im zeitlichen Rahmen des Praxisforschungsprojekts möglich, die steuerungsseitige Software-Entwicklung teilweise durchzuführen. Damit können die in Abschnitt~\ref{sec:task} formulierten Teilaufgaben als abgeschlossen betrachtet werden. Alles in Allem gehen die Umfänge zur vollständigen Umsetzung der Anlagenfunktionalität jedoch deutlich über diesen Beleg hinaus und müssen deshalb als nicht abgeschlossen angesehen werden. %Besonders da es sich um ein Non-Profit-Projekt handelt, ist es nur schwer möglich, ohne Erfahrung in der Arbeit mit dem Standard die Validierung nach BMW-Vorgabe in einem zeitlich und kostentechnisch vertretbaren Rahmen zu durchlaufen.
