%% Software

%%%% SUBFILE %%%%
% subfiles allows to compile from a subfile and not main document
% to make it work in a folder structure, the root document needs to be specified
% in each subfile with the following line (placed somewhere at the beginning):
% !TEX root =  ../document.tex

\chapter{Anlagensoftware}
\label{chap:software}

\section{Vorgehensweise}
\label{sec:software_procedure}

Wie für die meisten Entwicklungsschritte im Anlagenentstehungsprozess enthält der TMO V1 Standard auch für die Anlagensoftware eine Projektvorlage. Diese besteht im Wesentlichen aus zwei Teilen, einem Step7-Projekt, welches das Programm für die SPS enthält und einem TIA V13 SP1 Projekt für die Visualisierung. %In beiden Fällen handelt es sich um konsistente, kompilierbare und lauffähige Projekte. 
Die Projekte sind so aufgebaut, dass sie nahezu alle im Standard vorhandenen Bausteine und Konzepte implementieren. Die Nutzung dieser Vorlagen ist dabei nicht optional sondern obligatorisch. Wie in anderen Domänen des Standards ist auch bei der Anlagensoftware das Ziel, eine einheitliche Basis für ein breites Feld von Maschinen zu schaffen, auch wenn diese von unterschiedlichen Herstellern entwickelt werden.\par

Die Bearbeitung der Anlagensoftware orientiert sich am Leitfaden\citelothers{others_plc_sw} zur Entwicklung der S7-Software. Als Voraussetzung dienen der Ergebnisse der Planungsphase, sowie eine vorhandene Feinspezifikation. Folgende Vorgehensweise wird von BMW empfohlen:
\begin{enumerate}
    \itemsep 0.05em
    \item Vorbereitung
    \begin{enumerate}
        \itemsep 0.05em
        \item {Erstellen der Hardware-Konfiguration}\label{item:prep:hw_config}
        \item Erarbeiten der Symboltabellen
        \item Entfernen nicht benötigter Software Bausteine
    \end{enumerate} 
    \item Anbindung der HMI-Panels und Konfiguration der Funktionsgruppen (FG\abbrev{FG}{Funktionsgruppe})
    \item Konfiguration des Meldesystems
    \item Anlagendiagnose in der Steuerung
    \item {Entwicklung der Safety-Software}\label{item:prep:safety}
    \item Entwicklung der Logik für Aktorik und Sensorik
    \item Ablaufprogrammierung in Schrittketten
    \item Anbindung IT-Systemen\label{item:prep:it_systems}
\end{enumerate}

In den folgenden Abschnitten werden exemplarisch die Punkte~\ref{item:prep:hw_config}, \ref{item:prep:safety} und \ref{item:prep:it_systems} beschrieben. 

\section{Hardware-Konfiguration}
\label{sec:hw_config}
Basierend auf der in Abschnitt~\ref{sec:profinetList} beschriebenen PROFINET Liste wird die Hardware-Konfiguration vorgenommen. Zunächst wird die verwendete Steuerung eingefügt, benannt und mit der projektierten IP-Adresse versehen. Anschließend werden nach und nach alle Teilnehmer aus der PROFINET-Liste hinzugefügt, adressiert und die entsprechenden Ein-/Ausgangsbereiche zugeordnet. Danach wird das PROFINET Netzwerk konfiguriert. Das heißt, jedem Port, sofern er verschaltet werden soll, wird der Port zugeordnet, mit dem er verbunden ist. Dieses Vorgehen bietet erweiterte Diagnose-Möglichkeiten, wenn Teilnehmer am Netz ausfallen oder ausgetauscht werden. Für das Testprojekt entsteht die in Abbildung~\ref{fig:hw_config} dargestellte Hardware-Konfiguration. Die zugehörige PROFINET-Topologie ist in Abbildung~\ref{fig:profinet_topology} zu sehen.\par

\bild{tb}{width=0.85\textwidth}{hw_config}{Hardware-Konfiguration}

\bild{tb}{width=1\textwidth}{profinet_topology}{PROFINET Topologie}

\section{Safety-Software}
\label{sec:safety_software}

Als Grundlage für die Safety-Software dient die Safety-Matrix (Vgl. Abb.~\ref{fig:safety_matrix}). In dieser steht beschrieben, welches Nothalt-Organ welchen Lastkreis und welche Funktionsgruppe betrifft. Für den vergleichsweise geringen Umfang der Anlage im Testprojekt ist die Matrix recht simpel. Hier wirkt jedes Nothalt-Organ auf jede Funktionsgruppe. Bei komplexeren, flächenmäßig weitläufigen Anlagen lautet die Vorgabe, dass Nothalt-Organe jeweils nur Funktionsgruppen betreffen, die sich in unmittelbarer Nähe befinden. Speziell bei Nothalt-Tastern gilt, dass sich betroffene Funktionsgruppen im Sichtbereich der betätigenden Person befinden müssen.

\bild{tb}{width=0.4\textwidth}{safety_matrix}{Safety-Matrix}

\section{Anbindung IT-Systemen}
